%%%%%%%%%%%%%%%%%%%%%%%%%%%%%%%%%%%%%%%%%
% Wenneker Assignment
% LaTeX Template
% Version 2.0 (12/1/2019)
%
% This template originates from:
% http://www.LaTeXTemplates.com
%
% Authors:
% Vel (vel@LaTeXTemplates.com)
% Frits Wenneker
%
% License:
% CC BY-NC-SA 3.0 (http://creativecommons.org/licenses/by-nc-sa/3.0/)
% 
%%%%%%%%%%%%%%%%%%%%%%%%%%%%%%%%%%%%%%%%%

%----------------------------------------------------------------------------------------
%	PACKAGES AND OTHER DOCUMENT CONFIGURATIONS
%----------------------------------------------------------------------------------------

\documentclass[11pt]{scrartcl} % Font size

%%%%%%%%%%%%%%%%%%%%%%%%%%%%%%%%%%%%%%%%%
% Wenneker Assignment
% Structure Specification File
% Version 2.0 (12/1/2019)
%
% This template originates from:
% http://www.LaTeXTemplates.com
%
% Authors:
% Vel (vel@LaTeXTemplates.com)
% Frits Wenneker
%
% License:
% CC BY-NC-SA 3.0 (http://creativecommons.org/licenses/by-nc-sa/3.0/)
% 
%%%%%%%%%%%%%%%%%%%%%%%%%%%%%%%%%%%%%%%%%

%----------------------------------------------------------------------------------------
%	PACKAGES AND OTHER DOCUMENT CONFIGURATIONS
%----------------------------------------------------------------------------------------

\usepackage{amsmath, amsfonts, amsthm} % Math packages

\usepackage{listings} % Code listings, with syntax highlighting

\usepackage[english]{babel} % English language hyphenation

\usepackage{graphicx} % Required for inserting images
\graphicspath{{Figures/}{./}} % Specifies where to look for included images (trailing slash required)

\usepackage{booktabs} % Required for better horizontal rules in tables

\numberwithin{equation}{section} % Number equations within sections (i.e. 1.1, 1.2, 2.1, 2.2 instead of 1, 2, 3, 4)
\numberwithin{figure}{section} % Number figures within sections (i.e. 1.1, 1.2, 2.1, 2.2 instead of 1, 2, 3, 4)
\numberwithin{table}{section} % Number tables within sections (i.e. 1.1, 1.2, 2.1, 2.2 instead of 1, 2, 3, 4)

\setlength\parindent{0pt} % Removes all indentation from paragraphs

\usepackage{enumitem} % Required for list customisation
\setlist{noitemsep} % No spacing between list items

%----------------------------------------------------------------------------------------
%	DOCUMENT MARGINS
%----------------------------------------------------------------------------------------

\usepackage{geometry} % Required for adjusting page dimensions and margins

\geometry{
	paper=a4paper, % Paper size, change to letterpaper for US letter size
	top=2.5cm, % Top margin
	bottom=3cm, % Bottom margin
	left=3cm, % Left margin
	right=3cm, % Right margin
	headheight=0.75cm, % Header height
	footskip=1.5cm, % Space from the bottom margin to the baseline of the footer
	headsep=0.75cm, % Space from the top margin to the baseline of the header
	%showframe, % Uncomment to show how the type block is set on the page
}

%----------------------------------------------------------------------------------------
%	FONTS
%----------------------------------------------------------------------------------------

\usepackage[utf8]{inputenc} % Required for inputting international characters
\usepackage[T1]{fontenc} % Use 8-bit encoding

\usepackage{fourier} % Use the Adobe Utopia font for the document

%----------------------------------------------------------------------------------------
%	SECTION TITLES
%----------------------------------------------------------------------------------------

\usepackage{sectsty} % Allows customising section commands

\sectionfont{\vspace{6pt}\centering\normalfont\scshape} % \section{} styling
\subsectionfont{\normalfont\bfseries} % \subsection{} styling
\subsubsectionfont{\normalfont\itshape} % \subsubsection{} styling
\paragraphfont{\normalfont\scshape} % \paragraph{} styling

%----------------------------------------------------------------------------------------
%	HEADERS AND FOOTERS
%----------------------------------------------------------------------------------------

\usepackage{scrlayer-scrpage} % Required for customising headers and footers

\ohead*{} % Right header
\ihead*{} % Left header
\chead*{} % Centre header

\ofoot*{} % Right footer
\ifoot*{} % Left footer
\cfoot*{\pagemark} % Centre footer
 % Include the file specifying the document structure and custom commands
\usepackage{hyperref} % Include the hyperref package for URLs
\usepackage[backend=bibtex,style=numeric]{biblatex}
\usepackage{csquotes}
\usepackage{algorithm}
\usepackage{algpseudocode}
\addbibresource{Bibliography.bib} % Specify the bibliography file (ensure this file exists and contains entries)



%----------------------------------------------------------------------------------------
%	TITLE SECTION
%----------------------------------------------------------------------------------------

\title{	
	\normalfont\normalsize
	\textsc{University of Galway}\\ % Your university, school and/or department name(s)
	\vspace{25pt} % Whitespace
	\rule{\linewidth}{0.5pt}\\ % Thin top horizontal rule
	\vspace{20pt} % Whitespace
	{\huge  Project 1: Evolutionary Search (GAs)}\\ % The assignment title
	\vspace{12pt} % Whitespace
	\rule{\linewidth}{2pt}\\ % Thick bottom horizontal rule
	\vspace{12pt} % Whitespace
}

\author{\LARGE Cathal Lawlor} % Your name

\date{\normalsize\today} % Today's date (\today) or a custom date

\begin{document}

\maketitle % Print the title

\section{Github Repository}
Github repository with code \href{https://github.com/Laan33/ai_project_1}{here}

\section{Implementation details \& design choices}

\subsection{Selection method}
I implemented and tried roulette wheel selection, tournament selection and the monte carlo selection methods.
I found tournament to be the most reliable, and easiest to implement in python, to actually get consistent, repeatable results (that actually worked).

As I'll explain in the Potential improvements\ref{Potential improvements} section, I believe that tournament selection is the best choice for the travelling salesperson problem.
\subsection{Crossover methods}

I implemented both ordered crossover and partially mapped crossover. 

In partially mapped crossover (PMX)\cite{baeldung_pmx}, a random section of genes from one parent is copied to the child. The corresponding genes from the other parent are then mapped to avoid duplicates, ensuring each value appears exactly once. This method helps preserve relative positions within the crossover section.

In ordered crossover (OX)\cite{ordered_crossover_stackoverflow}, a random section of genes from one parent is copied to the child. The remaining positions are filled with genes from the other parent, in the order they appear, while skipping any duplicates. This ensures that the child inherits the relative ordering of cities from both parents.

I picked these two, as they both will work well for the travelling salesman problem, as they both preserve the order of the cities in the parents, which is important for the TSP.
The difference between the two is that PMX will preserve the relative order of the cities in the crossover section, while OX will not.

I chose these, as while PMX is more likely to produce a better child, OX can provide a more diverse population, which can be useful in the early stages of the algorithm.

\subsection{Mutation methods}

I implemented both swap mutation and inversion mutation.

In swap mutation, two random cities in the tour are selected, and their positions are exchanged. This introduces small, random changes while preserving the tour length.

In inversion mutation, a random subsection of the tour is selected, and the order of cities within that section is reversed. This helps the algorithm explore new solutions while maintaining most of the existing tour structure.


%------------------------------------------------





\section{Experimentation results \& analysis}
\subsection{Experimentation setup}

\subsection{Results}
\subsubsection{}


\section{Comparison with known optimal solutions}
\label{Comparison with known optimal solutions}

Comparing to the optimal solutions from Heidelberg university \cite{heidelberg_university_best_known}, my algorithm.

\begin{table}[h!]
\centering
\begin{tabular}{|c|c|c|c|}
\hline
\textbf{Dataset} & \textbf{My Algorithm} & \textbf{Optimal Solution} & \textbf{Difference} \\ \hline
\textbf{belin52} & 7970.9 & 7542 & 5.69\% \\ \hline
\textbf{kroA100} & Algorithm 2 & 21282 & 8\% \\ \hline
\textbf{pr1002} & Algorithm 3 & 259045 & 5\% \\ \hline
\end{tabular}
\caption{Comparison of algorithms with known optimal solutions}
\label{tab:comparison}
\end{table}


\section{Potential improvements}
\label{Potential improvements}
One easy, improvment I believe for my algorithm would be to employ elitism.
I implemented it in a early version of the program. I chose not to implement it, as I wanted to see how the various methods for crossover and mutation would handle the problem alone, without the help of elitism.
It would probably extract a small bit more performance out, especially for the bigger datasets, less so in the Berlin dataset.

I believe that tournament selection is a good choice, being the sweet spot for a genetic algorithm solving TSP. It's fast, simple, and maintains a good diversity and doesn't suffer with early convergence like roulette wheel selection can\cite{genetic_algorithm_afternoon}.
I also tried monte carlo selection, but I do not believe it works for the travelling salesperson problem.

\section{Conclusions}

\printbibliography
\end{document}
