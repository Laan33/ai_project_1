%%%%%%%%%%%%%%%%%%%%%%%%%%%%%%%%%%%%%%%%%
% Wenneker Assignment
% LaTeX Template
% Version 2.0 (12/1/2019)
%
% This template originates from:
% http://www.LaTeXTemplates.com
%
% Authors:
% Vel (vel@LaTeXTemplates.com)
% Frits Wenneker
%
% License:
% CC BY-NC-SA 3.0 (http://creativecommons.org/licenses/by-nc-sa/3.0/)
% 
%%%%%%%%%%%%%%%%%%%%%%%%%%%%%%%%%%%%%%%%%

%----------------------------------------------------------------------------------------
%	PACKAGES AND OTHER DOCUMENT CONFIGURATIONS
%----------------------------------------------------------------------------------------

\documentclass[11pt]{scrartcl} % Font size

\input{structure.tex} % Include the file specifying the document structure and custom commands
\usepackage{hyperref} % Include the hyperref package for URLs
\usepackage[backend=bibtex,style=numeric]{biblatex}
\usepackage{csquotes}
\usepackage{algorithm}
\usepackage{algpseudocode}
\addbibresource{Bibliography.bib} % Specify the bibliography file (ensure this file exists and contains entries)



%----------------------------------------------------------------------------------------
%	TITLE SECTION
%----------------------------------------------------------------------------------------

\title{	
	\normalfont\normalsize
	\textsc{University of Galway}\\ % Your university, school and/or department name(s)
	\vspace{25pt} % Whitespace
	\rule{\linewidth}{0.5pt}\\ % Thin top horizontal rule
	\vspace{20pt} % Whitespace
	{\huge  Project 1: Evolutionary Search (GAs)}\\ % The assignment title
	\vspace{12pt} % Whitespace
	\rule{\linewidth}{2pt}\\ % Thick bottom horizontal rule
	\vspace{12pt} % Whitespace
}

\author{\LARGE Cathal Lawlor} % Your name

\date{\normalsize\today} % Today's date (\today) or a custom date

\begin{document}

\maketitle % Print the title

% %----------------------------------------------------------------------------------------
% %	FIGURE EXAMPLE
% %----------------------------------------------------------------------------------------

% % \section{Image Interpretation}

% % \begin{figure}[h] % [h] forces the figure to be output where it is defined in the code (it suppresses floating)
% % 	\centering
% % 	\includegraphics[width=0.5\columnwidth]{swallow.jpg} % Example image
% % 	\caption{European swallow.}
% % \end{figure}

% %------------------------------------------------

% \subsection{What is the airspeed velocity of an unladen swallow?}

% While this question leaves out the crucial element of the geographic origin of the swallow, according to Jonathan Corum, an unladen European swallow maintains a cruising airspeed velocity of \textbf{11 metres per second}, or \textbf{24 miles an hour}. The velocity of the corresponding African swallows requires further research as kinematic data is severely lacking for these species.

\section{Github Repository}
Github repository with code \href{https://github.com/Laan33/ai_project_1}{here}

\section{Implementation details \& design choices}

\subsection{Selection method}
I implemented and tried roulette wheel selection, tournament selection and the monte carlo selection methods.
I found tournament to be the most reliable, and easiest to implement in python, to actually get consistent, repeatable results (that actually worked).

\subsection{Crossover methods}
\subsection{Crossover methods}

I implemented both ordered crossover and partially mapped crossover. 

For partially mapped crossover, I used the implementation from here \cite{baeldung_pmx}, where the algorithm is as follows:

\begin{algorithm}
\caption{Partially Mapped Crossover (PMX)}
\begin{algorithmic}
\State \textbf{Input:} Parent1, Parent2
\State \textbf{Output:} Child
\State Randomly select crossover segment [start, end]
\State Copy crossover segment from Parent1 to Child
\State Create mapping between crossover sections of Parent1 and Parent2
\For{each position in the Child}
    \If{position is not in crossover segment}
        \State value = Parent2[position]
        \While{value is already in Child}
            \State value = mapping[value]
        \EndWhile
        \State Child[position] = value
    \EndIf
\EndFor
\State \textbf{Return} Child
\end{algorithmic}
\end{algorithm}

For ordered crossover, I interpreted the algorithm from here \cite{ordered_crossover_stackoverflow}, where:

\begin{algorithm}
\caption{Ordered Crossover (OX)}
\begin{algorithmic}
\State \textbf{Input:} Parent1, Parent2
\State \textbf{Output:} Child
\State Randomly select crossover segment [start, end]
\State Copy crossover segment from Parent1 to Child
\State Create a list of remaining genes from Parent2 not in Child
\For{each position in Child}
    \If{position is empty}
        \State Fill position with the next gene from remaining list
    \EndIf
\EndFor
\State \textbf{Return} Child
\end{algorithmic}
\end{algorithm}

For ordered crossover, I interpreted the algorithm from here\cite{ordered_crossover_stackoverflow}, where:
\begin{verbatim}
	def ordered_crossover(parent1, parent2):
		size = len(parent1)
		# Select crossover points
		start, end = sorted(random.sample(range(size), 2))

		# Initialize offspring & copy the segment between the points from parent1
		child = [-1] * size 
		child[start:end] = parent1[start:end]
		
		# Get the remaining genes from parent2 & fill them in order
		remaining = [gene for gene in parent2 if gene not in child] 
		child = [remaining.pop(0) if gene == -1 else gene for gene in child] 
		return child
\end{verbatim}



\subsection{Mutation methods}
I implemented both swap mutation and inversion mutation.


%------------------------------------------------

%------------------------------------------------

% ...existing code...

% \bibliographystyle{unsrt}
% \bibliography{references}

\printbibliography

\end{document}
